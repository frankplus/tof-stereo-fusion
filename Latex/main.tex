\documentclass[12pt,a4paper,twoside,italian]{book} %twoside,
\raggedbottom % per evitare eccessivi spazi bianchi verticali

% Usare "oneside" invece di "twoside"
% nelle bozze, per risparmiare carta:
% "twoside" produce diverse pagine bianche
% alla fine dei capitoli.

\usepackage[utf8]{inputenc}
\usepackage[italian]{babel}
\usepackage[T1]{fontenc}
\usepackage{siunitx}

%interlinea 1.5
\linespread{1.5}

   %%%%%%%%%%%%%%%%%%%%%%%%%%%%%%%%%%%%%%%%%%%%%%%%%%%%%%%%%%%%
   % Se nella tesi si inseriscono dei passi in un'altra       %
   % lingua (inglese, per fissare le idee), si puo' istruire  %
   % il TeX di sillabare quella parte di testo con le regole  %
   % inglesi, invece che italiane. A questo scopo basta       %
   % scrivere                                                 %
   %                                                          %
   %    \documentclass[...,english,italian,...]{...}          %
   %                                                          %
   % al posto di \documentclass[...,italian,...],             %
   % dopodiche' la sillabazione sara' italiana fintanto che   %
   % non si incontra il comando \selectlanguage{english}.     %
   % Per tornare all'italiano si scrive                       %
   % \selectlanguage{italian}                                 %
   %%%%%%%%%%%%%%%%%%%%%%%%%%%%%%%%%%%%%%%%%%%%%%%%%%%%%%%%%%%%

% per la definizione dello stile di pagina e delle macro con le quali ridisegnare la pagina principale.
\usepackage{unisanniotesi}

% questo package è utilizzao per permettere l'utilizzo dell'ambiente algoritmic che consente di editare e formattare opportunamente frammenti di pseudocodice.
\usepackage{algorithmic}

%\usepackage{algorithm}  %algorithm serve per incapsulare algorithmic al fine di ottenere un oggetto flotable come una figura o una tabella.

% Col pacchetto tocbibind compariranno nell'indice anche
% la bibliografia ed eventualmente l'indice analitico
\usepackage[nottoc]{tocbibind}

% \usepackage{graphicx} % gia' caricato da unisanniotesi
\graphicspath{{./figure/}}
\usepackage{epstopdf}

% Per l'ipertesto:
% \usepackage{hyperref} % gia' caricato da unisanniotesi
\hypersetup{
  % pdfpagelayout=SinglePage, % default
  % pdfpagemode=UseOutlines, % default
  % bookmarksopen, % default
  % bookmarksopenlevel=2, % default;
  pdftitle={Fusione di dati Stereo e Time-of-Flight mediante tecniche di Deep Learning},
  pdfauthor={Francesco Pham},
  pdfsubject={Tesi sulla fusione di dati 3d da telecamere stereo e sensori Time-of-Flight mediante tecniche di Deep Learning},
  pdfkeywords={tesi laurea fusione 3d stereo time of flight deep learning}} 
%%%%%%%%%%%%%%%%%%%%%%%%%%%%%%%%%%%%%%%%%%%%%%%%%%%%%%%%%%%%

\usepackage{amsmath, amsfonts, amssymb, amsthm}
\usepackage{latexsym}

% altri pacchetti
\usepackage{float}
\usepackage{xcolor}
\usepackage{emptypage} % per nascondere la numerazione nelle pagine bianche
\usepackage{url}

\makeatletter
\renewcommand\chapter{\if@openright\cleardoublepage\else\clearpage\fi
                    \thispagestyle{empty}% original style: plain
                    \global\@topnum\z@
                    \@afterindentfalse
                    \secdef\@chapter\@schapter}
\makeatother


%%%%%%%%%%%%%%%%%%%%%%%%%%%%%%%%%%%%%%%%%%%%%%%%%%%%%%%
%                    graphicx                         %
%                                                     %
%   Uno dei pacchetti per l'inserzione di figure      %
%   in formato eps e` "graphicx". Ce ne sono diversi  %
%   altri da poter scegliere.                         %
%                                                     %
%   Esempio di uso: avendo un file di nome            %
%   figura1.eps questa si inserisce nella tesi        %
%   col comando                                       %
%                                                     %
%        \begin{figure}[ht]                           %
%        \begin{center}                               %
%        \includegraphics{figura1.eps}                %
%        \caption[nome breve]{nome lungo}             %
%        \label{etichetta}                            %
%        \end{center}                                 %
%        \end{figure}                                 %
%                                                     %
%   Il "nome breve" e` quello che apparira`           %
%   nell'indice delle figure ed e' opzionale.         %
%   Il "nome lungo" e' quello che appare              %
%   sotto la figura.                                  %
%   (Ci sono opzioni per scalare, spostare, ruotare   %
%   le figure).                                       %
%   Con \graphicspath{{./figure/}} si dice            %
%   al LaTeX di cercare le figure nella cartella      %
%   "figure" situata allo stesso livello di           %
%   questo documento                                  %
%                                                     %
%%%%%%%%%%%%%%%%%%%%%%%%%%%%%%%%%%%%%%%%%%%%%%%%%%%%%%%


   %%%%%%%%%%%%%%%%%%%%%%%%%%%%%%%%%%%%%%%%%%%
   %  Esempi di macro definite dall'utente.  %
   %  Le prime definiscono dei comandi per   %
   %  scrivere i caratteri speciali per      %
   %  gli insiemi numerici fondamentali      %
   %  (naturali, interi, razionali, reali,   %
   %  complessi                              %
   %%%%%%%%%%%%%%%%%%%%%%%%%%%%%%%%%%%%%%%%%%%

\newcommand{\N}{\mathbb{N}}
\newcommand{\Z}{\mathbb{Z}}
\newcommand{\Q}{\mathbb{Q}}
\newcommand{\R}{\mathbb{R}}
\newcommand{\C}{\mathbb{C}}

   %%%%%%%%%%%%%%%%%%%%%%%%%%%%%%%%%%%%%%%%%%%%
   %  Delle macro che definiscono operatori   %
   %  non predefiniti in LaTeX. Ogni utente   %
   %  aggiunge quelle che servono. Questi     %
   %  sono solo esempi arbitrari.             %
   %%%%%%%%%%%%%%%%%%%%%%%%%%%%%%%%%%%%%%%%%%%%

\DeclareMathOperator{\traccia}{tr}
\DeclareMathOperator{\sen}{sen}
\DeclareMathOperator{\arcsen}{arcsen}
\DeclareMathOperator*{\maxlim}{max\,lim}
\DeclareMathOperator*{\minlim}{min\,lim}
\DeclareMathOperator*{\deepinf}{\phantom{\makebox[0pt]{p}}inf}

    %%%%%%%%%%%%%%%%%%%%%%%%%%%%%%%%%%%%%%%%%%%%
    % Esempi di macro piu` elaborate,          %
    % contenenti degli argomenti.              %
    % Compongono gli indici delle sommatorie   %
    % e delle produttorie in modo diverso      %
    % da quello standard del TeX. Dovrebbero   %
    % funzionare bene quando gli estremi della %
    % sommatoria sono piccoli. Chi volesse     %
    % usarle estesamente farebbe bene a        %
    % lavorarci sopra.                         %
    %%%%%%%%%%%%%%%%%%%%%%%%%%%%%%%%%%%%%%%%%%%%

\newcommand{\varsum}[3]{\sum_{#2}^{#3}\!
   {\vphantom{\sum}}_{#1}\;}
\newcommand{\varprod}[3]{\sum_{#2}^{#3}\!
   {\vphantom{\sum}}_{#1}\;}

  %%%%%%%%%%%%%%%%%%%%%%%%%%%%%%%%%%%%%%%%%%%%%%%%%%%%%%%
  %          Numerazione delle formule                  %
  % Se non specificato altrimenti, il LaTeX numera le   %
  % formule come (capitolo.formula) (per esempio (2.5)  %
  % e` la quinta formula del secondo capitolo).         %
  % Con le istruzioni seguenti invece la numerazione    %
  % diventa (capitolo.sezione.formula) (per esempio     %
  % (3.2.6) e` la sesta formula della seconda sezione   %
  % del terzo capitolo):                                %
  %%%%%%%%%%%%%%%%%%%%%%%%%%%%%%%%%%%%%%%%%%%%%%%%%%%%%%%

%\makeatletter
%\@addtoreset{equation}{section}
%\makeatother
%\renewcommand{\theequation}%
%  {\thesection.\arabic{equation}}


              %%%%%%%%%%%%%%%%%%%%%%%%%%
              % Stile degli enunciati  %
              %%%%%%%%%%%%%%%%%%%%%%%%%%

%%%%%%%%%%%%%%%%%%%%%%%%%%%%%%%%%%%%%%%%%%%%%%%%%%%%%%%%%%%
% Con le dichiarazioni seguenti                           %
% teoremi, definizioni, proposizioni, lemmi e corollari   %
% vengono numerati capitolo per capitolo e con un         %
% contatore unico per tutti (per esempio, se subito dopo  %
% il Teorema 2.1 c'e' una definizione, questa sara'       %
% Definizione 2.2)                                        %
%%%%%%%%%%%%%%%%%%%%%%%%%%%%%%%%%%%%%%%%%%%%%%%%%%%%%%%%%%%

\theoremstyle{plain}
\newtheorem{teorema}{Teorema}[chapter]
\newtheorem{proposizione}[teorema]{Proposizione}
\newtheorem{lemma}[teorema]{Lemma}
\newtheorem{corollario}[teorema]{Corollario}

\theoremstyle{definition}
\newtheorem{definizione}[teorema]{Definizione}
\newtheorem{esempio}[teorema]{Esempio}

\theoremstyle{remark}
\newtheorem{osservazione}[teorema]{Osservazione}

  %%%%%%%%%%%%%%%%%%%%%%%%%%%%%%%%%%%%%%%%%%%%%%%%%%%%%%%%
  % I comandi si usano cosi`:                            %
  %                                                      %
  %   \begin{teorema}[di Pitagora]                       %
  %   La somma dei quadrati ecc.                         %
  %   \end{teorema}                                      %
  %                                                      %
  % Le parole "di Pitagora" fra parentesi quadre         %
  % sono facoltative. Non bisogna inserire               %
  % manualmente degli spazi prima e dopo gli enunciati,  %
  % perche' e` automatico!                               %
  %%%%%%%%%%%%%%%%%%%%%%%%%%%%%%%%%%%%%%%%%%%%%%%%%%%%%%%%


  %%%%%%%%%%%%%%%%%%%%%%%%%%%%%%%%%%%%%%%%%%%%%%%%%%%%%%%%%%%%%%
  % Il pacchetto amsthm definisce anche l'ambiente "proof"     %
  % per le dimostrazioni.                                      %
  % Esempio di uso:                                            %
  %                                                            %
  %   \begin{proof}                                            %
  %   Sia $X$ un insieme ecc.                                  %
  %   \end{proof}                                              %
  %                                                            %
  %%%%%%%%%%%%%%%%%%%%%%%%%%%%%%%%%%%%%%%%%%%%%%%%%%%%%%%%%%%%%%

       %%%%%%%%%%%%%%%%%%%%%%%%%%%%%%%%%%%%%%%%%%%%%%%%%%%%%%%
       %                   makeidx                           %
       %                                                     %
       % Pacchetto per la generazione automatica dell'indice %
       % analitico. Per esempio, se vogliamo che la parola   %
       % "analitico" venga indicizzata nella frase           %
       %                                                     %
       %    "un metodo analitico di soluzione"               %
       %                                                     %
       % bisogna scrivere                                    %
       %                                                     %
       %    "un metodo analitico\index{analitico} di         %
       %              soluzione".                            %
       %                                                     %
       % Compilando il file, il LaTeX produrra' un file      %
       % ausiliario che termina con ".idx". Bisogna far      %
       % processare questo file idx dal programma            %
       % ausiliario "bibtex", che produrra' a sua volta un   %
       % altro file ancora. Dare infine un'ultima passata    %
       % col LaTeX. Si puo' tranquillamente lasciare         %
       % la compilazione dell'indice verso la fine della     %
       % stesura del lavoro, quando tutto e' ormai quasi     %
       % definitivo.                                         %
       %                                                     %
       %%%%%%%%%%%%%%%%%%%%%%%%%%%%%%%%%%%%%%%%%%%%%%%%%%%%%%%

\usepackage{makeidx}
\makeindex

% Ridefiniamo la riga di testa delle pagine:
\usepackage{fancyhdr}
\pagestyle{fancy}

% per la documentazione on line sul package fancyhdr: http://www.ctan.org/tex-archive/info/italian/fancyhdr/itfancyhdr.pdf

\usepackage{fancyhdr}
\pagestyle{fancy}
\renewcommand{\chaptermark}[1]{\markboth{#1}{}}
\renewcommand{\sectionmark}[1]{\markright{\thesection\ #1}}
\fancyhf{}
\fancyhead[LE,RO]{\bfseries\thepage}
\fancyhead[LO]{\bfseries\rightmark}
\fancyhead[RE]{\bfseries\leftmark}
\renewcommand{\headrulewidth}{0.5pt}
\renewcommand{\footrulewidth}{0pt}
\setlength{\headheight}{14.5pt}

               %%%%%%%%%%%%%%%%%%%%%%%%%%%%%%%%%%%%%%
               %  Informazioni generali sulla Tesi  %
               %    da usare nell'intestazione      %
               %%%%%%%%%%%%%%%%%%%%%%%%%%%%%%%%%%%%%%

% definiamo una serie di macro generali che poi potranno essere riutilizzate.
  \titolo{Fusione di dati \\Stereo e Time-of-Flight mediante \\tecniche di Deep Learning}
  \laureando{Francesco Pham}
  \annoaccademico{2018-2019}
  \datalaurea{25 SETTEMBRE 2019}
  \facolta{Ingegneria dell'Informazione} % (default)
%  \corsodilaurea{NomeCorsoDiLaurea} % per la laurea vecchio ordinamento
 \corsodilaureatriennale{Ingegneria Informatica}
% \corsodilaureaspecialistica{NomeCorsoDiLaureaSpecialistica}
  \relatore[Prof.]{Pietro Zanuttigh}
	
	% in questo caso la macro usa di default il titolo di prof... per altri usi utilizzare []
  \correlatore[Ing.]{Gianluca Agresti}
% \correlatoreDue{Secondo Correlatore}
  \dedica{Ai miei genitori, \\che mi hanno sempre sostenuto.} % (facoltativo)


   %%%                                    %        %%    %%
  %   %                                   %         %     %
  %      %%%  % %%  %%%%   %%%         %%%%  %%%    %     %    %%%%
  %     %   % %%  % %   % %   %       %   % %   %   %     %   %   %
  %     %   % %     %   % %   %       %   % %%%%%   %     %   %   %
  %   % %   % %     %   % %   %       %   % %       %     %   %  %%
   %%%   %%%  %     %%%%   %%%         %%%%  %%%%  %%%   %%%   %% %
                    %
                    %


                          %%%%%               %
                            %
                            %    %%%   %%%%  %%
                            %   %   % %       %
                            %   %%%%%  %%%    %
                            %   %         %   %
                            %    %%%% %%%%   %%%


\begin{document}

         %%%%%%%%%%%%%%%%%%%%%%%%%%%%%%%%%%%%%%%%%%%%%%%%%
         %                 Intestazione                  %
         %                                               %
         % Per l'intestazione completa bisogna  essersi  %
         % procurati il file "unisannioLogo.eps"         %
         %%%%%%%%%%%%%%%%%%%%%%%%%%%%%%%%%%%%%%%%%%%%%%%%%

\frontmatter
\maketitle

  %%%%%%%%%%%%%%%%%%%%%%%%%%%%%%%%%%%%%%%%%%%%%%%%%%%%%%%%%%%
  %   Si puo` scegliere fra scrivere tutta la tesi in un    %
  %   solo file, oppure distribuire ogni capitolo in un     %
  %   file a parte. Qui si e` scelto tenere separati i      %
  %   vari capitoli, che vengono caricati con \include      %
  %%%%%%%%%%%%%%%%%%%%%%%%%%%%%%%%%%%%%%%%%%%%%%%%%%%%%%%%%%%

\renewcommand{\theequation}{\arabic{equation}}%consigliato per migliorare i numeri di equazione nell'introduzione
\renewcommand{\thesection}{\arabic{section}}  %consigliato per migliorare i numeri di equazione nell'introduzione


%\input{abstract}
%\input{introduzione}

\tableofcontents
%\listoffigures

\mainmatter

\renewcommand{\theequation}{\arabic{chapter}.\arabic{equation}} %si torna alle formule numerate come da default
\renewcommand{\thesection}{\arabic{chapter}.\arabic{section}} %consigliato per migliorare i numeri di equazione nell'introduzione

\chapter{Introduzione}  
La stima della profondità ha da sempre rappresentato un problema di massimo interesse. L'informazione sulla profondità è importante, ed in alcuni casi essenziale, per molteplici applicazioni pratiche della visione artificiale: guida autonoma, robotica, ricostruzione 3D e realtà aumentata sono soltanto alcune. \\
L'acquisizione delle geometrie tridimensionali di scene del mondo reale rappresenta da sempre un problema complesso, e gli strumenti sono stati accessibili soltanto in grosse compagnie e centri di ricerca. Nel corso degli anni, molte tecniche sono state sviluppate e nuovi dispositivi, dai costi più ridotti, sono stati introdotti nel mercato.\\
Le principali tecnologie per la percezione dell'ambiente sfruttano sensori ottici che catturano la luce visibile o infrarossa che viene riflessa dalla scena. Questo lavoro di tesi tratta due categorie di sensori in particolare, entrambi molto diffusi e accessibili: i sensori a visione stereoscopica e i sensori basati sul tempo di volo (Time-Of-Flight, ToF).\\
Nello stesso periodo, nel mondo della computer vision, il machine learning (e deep learning in particolare) si è dimostrato uno strumento che ha ampiamente incrementato le prestazioni rispetto ad altri metodi deterministici, in molti problemi ritenuti complessi, fino a superare le prestazioni umane. E proprio grazie al machine learning è stato possibile sviluppare sistemi efficaci per la ricostruzione delle informazioni catturate dai sensori.

\section{Descrizione del progetto}
L'obiettivo del lavoro di questa tesi è lo sviluppo di un sistema di machine learning per la fusione dei dati tridimensionali forniti dai sensori, cercando di fornire una ricostruzione più accurata. La scelta dei sensori gioca perciò un ruolo fondamentale: le telecamere stereo e il sensore ToF tendono ad avere caratteristiche complementari, e la loro combinazione si è rivelato efficace negli ultimi anni. 

\section{Principi di funzionamento del sistema di acquisizione}
Il sistema di acquisizione è composto da una coppia di telecamere stereo e da un sensore Time-of-Flight disposti adiacentemente. I due dispositivi catturano la stessa scena allo stesso istante.

\begin{figure}[ht]
    \centering
    \includegraphics[width=0.5\columnwidth]{tof_stereo_acquisition_system}
    \caption[Sistema di acquisizione ToF-Stereo]{Rappresentazione del sistema di acquisizione ToF-Stereo. Il sensore ToF è posizionato sotto la telecamera di riferimento della coppia stereo.}
    \label{sistema_di_acquisizione}
\end{figure}

\subsection{Sistema di visione stereo}
Il sensore a visione stereoscopica consiste nell'acquisire due immagini bidimensionali da una coppia di telecamere rettificate, ossia allineate lungo un asse che inquadrano la stessa scena. Lo stesso punto P dello spazio viene proiettato nel piano dell'immagine di ciascuna delle telecamere. I punti risultanti \(p_L\) e \(p_R\), detti \textit{omologhi}, hanno le stesse coordinate verticali, considerata la rettificazione dei sensori. Viene chiamata \textit{disparità} lo scostamento tra le coordinate orizzontali: \[d = u_L - u_R\] Tramite questo valore è possibile determinare la posizione del punto P nello spazio.\\
Uno dei principali vantaggi è il costo ridotto delle telecamere CCD o CMOS, facilmente accessibili nel mercato. Tale sensore è inoltre passivo, quindi può sfruttare l’illuminazione dell’ambiente. Ciò permette di essere utilizzato in ambienti esterni, al contrario di altri sensori che sfruttano
pattern proiettati con illuminatori. Inoltre può avere alte risoluzioni e una limitata quantità di rumore.\\
Lo svantaggio maggiore è la dipendenza dei risultati forniti da questi sensori dalla tessitura delle immagini utilizzata nel calcolo degli omologhi, ad esempio sono evidenti delle difficoltà nell'analisi di scene con pattern uniformi o ripetitivi. Pertanto essi presentano un’accuratezza solitamente limitata.

\begin{figure}[ht]
    \centering
    \includegraphics[width=0.5\columnwidth]{ZED}
    \caption[Sensore di visione stereo]{Sensore stereo ZED}
    \label{zed}
\end{figure}

\subsection{Sensore Time-Of-Flight}
Il sensore ToF determina le informazioni riguardo la profondità sulla base del fatto che le onde elettromagnetiche viaggiano alla velocità \(c\approx3\times 10^8[m/s]\). Il sensore emette un impulso luminoso (tipicamente infrarosso) il quale colpisce la scena e viene riflesso indietro. Viene dunque misurato il tempo $\tau$ che occorre al segnale luminoso per percorrere un tragitto pari al doppio della distanza dell'oggetto dalla telecamera $\rho$. La relazione che lega $\rho$ e $\tau$ è $$\rho=\frac{c\tau}{2}$$
Per misurare il tempo cercato e migliorare la risoluzione del sensore, l'impulso emesso è modulato secondo un segnale sinusoidale, così che anche l'eco abbia lo stesso andamento, ma
sfasato e attenuato in ampiezza.
\begin{figure}[ht]
    \centering
    \includegraphics[width=0.4\columnwidth]{time-of-flight}
    \caption[Principio time-of-flight]{Segnale sinusoidale trasmesso e corrispondente segnale ricevuto da un sensore ToF}
    \label{tof}
\end{figure}\\
Dispositivi ToF costituiti da un singolo trasmettitore e un singolo ricevitore, vengono tipicamente utilizzati in telemetri laser. Al fine di realizzare una mappa bidimensionale dell'ambiente, un sistema è quello di montare il sensore su una piattaforma roteante, tale configurazione ha trovato l'utilizzo nel campo dell'automobilistica. Il sistema utilizzato in questo lavoro di tesi appartiene bensì alla famiglia dei sensori ToF scanner-less chiamati anche ToF camera.\\
Nelle ToF camera, gli impulsi luminosi sono segnali infrarossi inviati tramite LED e il ricevitore è una matrice di sensori CCD/CMOS. Diversamente dal meccanismo di tipo scanner, le ToF camera catturano le geometrie della scena in un singolo scatto nella quale ogni pixel misura, indipendentemente dalle altre, la distanza del punto della scena di fronte.\\
I vantaggi maggiori sono, al contrario delle telecamere stereo, l'indipendenza dal contenuto della scena, e la maggiore rapidità poichè non richiede algoritmi particolari per il calcolo della disparità. \\
Tuttavia, la tecnologia non è esente da alcune limitazioni: 
\begin{itemize}
    \item Per superfici poco riflettenti o di colore scuro, il segnale di ritorno è debole, ciò risulta in misure poco accurate. 
    \item La limitata risoluzione spaziale non permette di ricavare una informazione dettagliata della geometria della scena.
    \item Il "multipath error" è un ulteriore problema che appare tipicamente vicino alle zone di incidenza tra le superfici. Esso è provocato dalla riflessione multipla del segnale luminoso prima di raggiungere il sensore. 
\end{itemize}

\section{Il Machine Learning}
Il machine learning, o apprendimento automatico, è una branca dell'informatica che fornisce ai computer la capacità di imparare ad eseguire un task da una certa esperienza, senza essere esplicitamente programmati a farlo. In sostanza, il machine learning esplora l'utilizzo di metodi matematico-computazionali per apprendere informazioni direttamente dai dati, senza modelli matematici ed equazioni predeterminate. Gli algoritmi di apprendimento automatico migliorano le loro prestazioni in modo "adattivo" mano a mano che gli "esempi" da cui apprendere aumentano. \\
I problemi che il machine learning punta a risolvere vengono classificati in tre categorie:
\begin{itemize}
    \item \textbf{Apprendimento supervisionato}: ogni istanza del set di esempi (training set) presenta gli input e i corrispondenti valori attesi in output. Questi esempi vengono presentati uno per volta alla macchina, il quale deve essere in grado di approssimare l'esatta natura della relazione presente tra l'input e il corrispondente output. L’obiettivo finale è dunque quello di insegnare al modello la capacità di predire correttamente i valori attesi su un set di istanze non presenti nel training set (test set). Questo scenario è quello trattato in questo lavoro di tesi.
    \item \textbf{Apprendimento non supervisionato}: al contrario dell'apprendimento supervisionato, gli output corrispondenti agli input forniti non sono conosciuti. All'algoritmo viene presentato bensì numerosi dati dai quali la macchina deve estrarre le caratteristiche o i pattern nascosti. 
    \item \textbf{Apprendimento per rinforzo}: si basa sul presupposto di potere ricevere degli stimoli dall'esterno a seconda delle scelte dell'algoritmo. Contrariamente all’apprendimento supervisionato nella quale si conosce l’uscita corretta che il sistema deve dare, l’apprendimento per rinforzo viene adoperato quando si ha a disposizione solamente di un’informazione qualitativa (giusto/sbagliato, successo/fallimento), chiamata segnale di rinforzo.
\end{itemize}

\subsection{Reti neurali artificiali}

\subsection{Reti neurali convoluzionali}
\chapter{Acquisizione dei dati}
Come già accennato, l'addestramento delle reti neurali convoluzionali richiede un dataset. Il training set deve essere sufficientemente grande in modo che la rete possa generalizzare in modo appropriato, cioè dare risultati plausibili per input che non ha mai visto. Al momento, tuttavia, l'acquisizione accurata di dati ground truth per scene 3D reali non è una operazione immediata. Per questa ragione, in questo lavoro verrà utilizzato un dataset sintetico simulato con \textit{Blender} composto da 55 scene 3D differenti. Le scene contengono diverse problematiche di acquisizione che includono riflessi, diversi tipi di illuminazione, e pattern ripetitivi. 

\section{Preparazione dei dati}
Per poter compiere una corretta fusione delle informazioni tridimensionali del ToF camera e del sensore stereo, bisogna fare in modo che i dati forniti siano all'interno dello stesso sistema di riferimento.

\paragraph{Interpolazione e riproiezione}
Il sensore Time-of-Flight fornisce una mappa di profondità della scena. Per poter eseguire la fusione con i dati restituiti dal sensore stereo, bisogna tenere conto le diverse risoluzioni. Infatti, il sensore ToF ha una risoluzione nettamente inferiore rispetto alle moderne telecamere a colori, perciò è necessario interpolare le immagini restituite in modo da portarlo alla risoluzione delle telecamere stereo.\\
Inoltre, le informazioni sulla profondità del sensore ToF vengono riproiettate sulla prospettiva della telecamera a colori di riferimento che, senza perdita di generalità, assumiamo essere la telecamera destra della coppia stereo. 

\paragraph{Calcolo della disparità}
Il sensore Time-of-Flight e il sensore stereo rappresentano la scena 3D in modo differente. Come già visto, il sistema stereo ritorna una mappa di disparità dove ogni pixel rappresenta la differenza di posizione tra i punti omologhi nelle due immagini. La ToF camera d'altro canto ritorna una mappa di profondità nella quale ogni pixel esprime la distanza tra il sensore e il corrispondente punto sulla scena. La disparità e la profondità sono legate da una relazione di proporzionalità inversa:
$$z=\frac{b\cdot f}{d}$$
dove $z$ è la distanza, $d$ è la disparità, $b$ è la baseline dello stereo e $f$ è la lunghezza focale delle telecamere \cite{rif1}. La disparità è stata scelta come unità di misura comune, dunque la mappa di profondità del ToF viene convertita in una mappa di disparità in modo tale da rendere omogenee le rappresentazioni dei due sensori.

\paragraph{Data augmentation}
Per \textit{data augmentation} si intende una tecnica che consiste nell'apportare alcune modifiche casuali nelle istanze del dataset, come rotazioni, ritagli, traslazioni e altre operazioni di image processing. Lo scopo è quello di ampliare il numero di esempi nel training set per una maggiore robustezza e variabilità, e di conseguenza ridurre le probabilità di overfitting.\\
Per questo lavoro sono stati effettuati casualmente operazioni di ritaglio, ribaltazione orizzontale e verticale dell'immagine e rotazione di $\pm \ang{5}$. Il prodotto di queste operazioni è un training set composto da 3902 esempi di dimensioni 128x128.
\chapter{Fusione tramite deep learning}

\section{Le mappe in input}

\section{Convoluzione dilatata}

\section{Reti neurali residuali}

\section{Confronti tra le varie architetture}

\section{La rete neurale selezionata}
\chapter{Analisi dei risultati}

\section{Tensorflow}

\section{Il processo di training}

\section{Valutazione dei risultati}

\subsection{Valutazione sul dataset sintetico}

\subsection{Valutazione sul dataset reale}
\chapter{Conclusioni}
La sfida all'origine di questa tesi è stata quella di costruire un sistema in grado di fondere i dati tridimensionali forniti da due diversi tipi di sensori, ossia un sensore a tempo di volo e una coppia di telecamere stereo. I risultati riportati hanno mostrato come l'utilizzo di tecniche di deep learning all'avanguardia, permetta di realizzare un modello in grado di massimizzare le informazioni fornite dai due sensori, realizzando una ricostruzione più accurata delle strutture tridimensionali della scena catturata. In particolare si è visto come l'utilizzo delle reti neurali residuali assieme alle convoluzioni dilatate abbia effettivamente apportato benefici tangibili sulle performance della rete nella stima della disparità.\\
I risultati ottenuti da questo lavoro di tesi possono inoltre essere applicati per la fusione di una diversa combinazione di sensori.


%\appendix

%\input{appendici}

\backmatter

%\addcontentsline{toc}{chapter}{Bibliografia}
\nocite{*}
\bibliographystyle{plain}
\bibliography{tesi}

% si può scegliere di utilizzare direttamente l'imput di un file 'biblio.tex' da editare a mano.
%\input{biblio}

\printindex % se si fa l'indice analitico.

\end{document}

