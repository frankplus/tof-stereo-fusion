\chapter{Fusione tramite deep learning}

\section{Confronti tra le varie architetture}
Partiamo dalla seguente architettura, il quale prende come input le due mappe di disparità dai corrispettivi sensori. 
\begin{figure}[ht]
    \centering
    \includegraphics[width=0.8\columnwidth]{noamp_4conv_8filters}
    \caption[Architettura 4 convoluzioni 8 filtri]{Una prima architettura della CNN}
\end{figure}\\
L'ultimo strato di convoluzione deve avere necessariamente un filtro in quanto deve generare una singola mappa di disparità corrispondente alla fusione degli input.\\
La CNN è costituita da una sequenza di blocchi, dove ciascun blocco ha uno strato di convoluzione, uno di \textit{batch normalization} e infine uno strato di rettificazione lineare (\textit{ReLU}).
\paragraph{Batch Normalization}
Il batch normalization esegue una normalizzazione del mini-batch ad ogni sub-strato della rete. L'utilizzo del batch normalization ha l'intento di velocizzare il training della rete neurale e migliorarne la stabilità e le performance. \\


\section{Convoluzione dilatata}

\section{Introduzione della ampiezza del segnale ToF}

\section{Reti neurali residuali}

\section{La rete neurale selezionata}