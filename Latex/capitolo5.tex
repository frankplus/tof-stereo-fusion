\chapter{Conclusioni}
La sfida all'origine di questa tesi è stata quella di costruire un sistema in grado di fondere i dati tridimensionali forniti da due diversi tipi di sensori, ossia un sensore a tempo di volo e una coppia di telecamere stereo. I risultati riportati hanno mostrato come l'utilizzo di tecniche di deep learning all'avanguardia, permetta di realizzare un modello in grado di massimizzare le informazioni fornite dai due sensori, realizzando una ricostruzione più accurata delle strutture tridimensionali della scena catturata. In particolare si è visto come l'utilizzo delle reti neurali residuali assieme alle convoluzioni dilatate abbia effettivamente apportato benefici tangibili sulle performance della rete nella stima della disparità.\\
I risultati ottenuti da questo lavoro di tesi possono inoltre essere applicati per la fusione di una diversa combinazione di sensori.